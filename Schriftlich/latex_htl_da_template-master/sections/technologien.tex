

\section{ESP32:}


\textbf{}
\begin{itemize}
    \item Der ESP32 unterstützt bis zu 4 × 16 MB externen Speicher.
    \item Man kann den internen Takt (8 MHz) oder einen
    externen Quarztakt mit üblicherweise 160 MHz nutzen. 
    Wenn man den Prozessor zurück setzt übernimmt er auch das System-Timing.
    \item Hall-Sensor: Kann für die Messung von Magnetfeldschwankungen genutzt werden.
    \item Ein interner Temperatursensor mit einem Messbereich von –40 bis 125 Grad ist ebenfalls vorhanden.
    Die analogen Messwerte werden wie bei dem Hall-Sensor
    von einem Analog-Digital-Wandler digitalisiert. 
    In neueren Modellen ist kein Temperatur Sensor mehr verbaut.
    \item Es sind 34 GPIOs (universelle Ein- und Ausgänge) vorhanden. 
    Verwendet werden diese für Ein-und Ausgabe analoger und digitaler Signale. 
    Die Pins sind mehrfach belegt. 
    Mit internen Pull-down und Pull-up-Widerständen können definierte Zustände herbeigeführt werden.
    \item Der ESP32 kann Signale von bis zu zehn unterschiedlichen TouchSensoren verarbeiten. 
    \item Der ESP32 Unterstützt Bluetooth als auch W-Lan im 2.4-GHz Bereich und kann diese 
    Empfangen und Senden. 
    \item Das WLAN hat einen IEE 802.11 Standard.
    \item Es wird Bluetooth 4.2 und Bluetooth low-energy unterstützt.
    \item UART (Universal Asynchronous Receiver Transmitter) 
    \item Impulse können mit acht Impulszählern erfasst werden. 
    \item Der ESP32 hat vier 64-Bit-Universaltimer. Diese kann man via Software steuern.
    \item Es gibt Watchdog-Timer. Man unterscheidet zwischen Main-Watchdog-Timern und RTC-Watchdog-Timern. 
    Auslösen kann man hiermit einen CPU-Reset , ein Core-Reset oder ein Interrupt.
    \item  12-Bit-A/D-Wandler (Analog-Digital-Wandler) mit 18 Kanälen 
    \item  8-Bit-DAC (Digital-Analog-Wandler)
    \item SPI-Schnittstellen (SPI1, HSPI and VSPI) mit Master- oder Slave-Modus
    \item Es werden zwei I2C-Bus-Schnittstellen vorgehalten, die im Master- oder SlaveModus betrieben werden können.
    \item Pulsweitenmodulation (PWM) um Geräte wie Motoren, elektrische Heizungen oder Ähnliches zu steuern. 
    \item Ein Infrarot-Controller, mit 8 programmierbaren Kanälen.
    
\end{itemize}

