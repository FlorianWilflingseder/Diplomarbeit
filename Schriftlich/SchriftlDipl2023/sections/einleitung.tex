\section{Sinn des Projektes:}
Das Projekt Pool-Überwachung soll IoT-Liebhabern ermöglichen jegliche Art von stillen Gewässern zu überwachen und dies alles auf ihrem Android, IOS, Windows, Linux Geräten oder in einer Webanwendung einfach einzusehen. 
Das Projekt ermöglicht es den Benutzern einen Schwimmer ins Wasser zu legen wie bei einem einfachen Pool-Thermometer. 
Jedoch kann dieser nicht nur die Temperatur anzeigen, sondern verfügt über die Funktion den PH-Wert und den NTU-Trübungswert zu erkennen. Ein zusätzliches Feature welches, vielen mit noch zu beaufsichtigenden Kindern ein Gefühl von Absicherung geben kann, ist eine Wellenerkennung die einen Alarm an das Handy sendet, falls ein verbauter Beschleunigungssensor verdächtige Aktivitäten wahrnimmt. 
Man kann diese Daten alle gemütlich in seiner App sehen und sich auch den Verlauf der Werte über bestimmte Zeiträume anzeigen lassen, dies kann wichtig sein falls nicht genug Chlor im Wasser ist oder nicht genügend reinigende Pflanzen in einem Teich gepflanzt sind. Da das Verbauen solcher Instrumente in das Pool sehr teuer ist oder im Nachhinein schwer möglich ist kommt Pool-Überwachung ins Spiel. 
\newline
Angefangen wurde mit einer Marktanalyse, die die Mitbewerber in diesem Markt veranschaulicht, die Vor- und Nachteile heraushebt und den genauen Markt definiert.

\section{Marktanalyse:}
Das Produkt richtet sich an Privatpersonen 
die einen Garten mit einer beliebigen Schwimmanlage besitzen. 
Die Altersgruppe wird sich an alle Geschlechter zwischen 25-60 Jahren richten. 
Das Produkt wird besonders attraktiv für Erwachsene 
mit Kindern sein die noch Beaufsichtigung benötigen wenn sie sich
in der Nähe von Gewässern aufhalten, wegen der Wellenerkennungsfunktion.
\section{Am Markt befindliche Produkte:}
\textbf{Smart Poolthermometer Starter Set:}


Das Set besteht aus 3 Teilen, einem schwimmenden Poolthermometer, 
einem Aussenfühler und einem Netzwerk Gateway. Kommt mit einer eigenen Smart App dazu. 
Es kann die Wassertemperatur die Luftaußentemperatur und die Luftfeuchtigkeit gemessen werden.
Netzwerkfähig (funkt das Gateway (bei freier Sichtline bis zu 100 Meter) an, 
der mit dem W-Lan Verbunden ist). Unterstützt Amazon Alexa und Conrad Connect. 
Wie der Pool so auch der Außenthermometer werden mit 2x AA Batterien betrieben. 
Der Außenthermometer ist dazu nicht Wetterfest und es muss eine extra Schutzhülle 
mitbestellt werden. Details über die einzelnen Bauteile werden nicht bekannt gegeben.


\textbf{WLAN Swimming Pool Thermometer Bundle|Inkbird:}


Hat ein Aufstellbaren-Display, Outdoor Hygrometer und Thermometer, 
Datenlogger, Export-Funktion, Cloud, App. 
Es kann die Wassertemperatur die Luftaußentemperatur und die Luftfeuchtigkeit gemessen werden.
Es wird nur ein Display zur Anzeige mitgeliefert und man 
kann seine Daten 12 Monate kostenlos in einer Cloud speichern.
Details über die einzelnen Bauteile werden nicht bekannt gegeben.

\textbf{Elektrobock ELBO-073 Wireless Pool Alarm:}


Dieses Gerät dient nur als Alarmanlage für das Wasser. 
Es erkennt den Wellengang und löst seine eingebaute Sirene aus. 
Das Gerät besitz ein Gegenstück welchen man in seiner Nähe an einer Steckdose anstecken 
kann und ein Signal von dem Sender im Wasser bekommt. 
Dadurch läutet die Sirene an 2 Standorten. Ansonsten bietet dieses keine zusätzlichen Funktionen.
Details über die einzelnen Bauteile werden nicht bekannt gegeben.
\newpage
\section{Grundkonzept:}
Man lässt die Sendereinheit im Wasser schwimmen und stellt die Empfängereinheit an einem trockenen Ort mit Strom- und W-Lan-Zugang auf. 
Nach dem beide Geräte aktiviert worden sind fangen sie mit dem Monitoring an und überwachen das Gewässer. 
Die Empfängereinheit erhält via Funk vom Sender die Daten und speichert diese ab. Auf diese Daten kann man dann mit seinem Smartphone oder Computer mit der dazugehörigen App zugreifen und sich die Daten anzeigen lassen. 
Der verbaute Alarm lässt sich in der App aktivieren und deaktivieren falls dieser nicht benötigt wird um 
unnötige Benachrichtigungen zu vermeiden.
Der Sender sendet alle 3 Stunden die Daten an die Empfängereinheit um ein regelmäßiges und vergleichbares Bild der aufgenommenen Daten zu ermöglichen.
Wenn der Sender inaktiv ist befindet er sich im Stromsparmodus um strom zu sparen.
\section{Umfrage:}
Im Zuge der Diplomarbeit wurden Freunde Verwandte und Bekannte zum Design des Frontends 
befragt. Die Fragen bezogen sich auf das Design des Frontends, welche Daten angezeigt werden sollen,
den Zeitraum den die Grafiken abdecken sollen und welche Funktionen in welcher Ausführung gefragt sind.

Zum Backend wurde besprochen welche wie lange die Daten gespeichert werden sollen. Ob es nötig ist diese über mehrere Jahre zu speichern
und ob Sie immer Abrufbar sein sollen oder möglicherweise als CSV-Datei bereitgestellt werden sollen. 
Aus den Antworten hat sich dann der das jetztige Design und der Aufbau des Projektes ergeben.
\newpage
\section{Key-Features:}
Die Key-Features dieses Produktes sind:
\begin{itemize}
    \item Temperatur (Messung der Temperatur)
    \item Ph-Wert (Messung des Ph-Wertes)
    \item Trübungseinheit-NTU (Messung des NTU-Wertes) 
    \item Alarm bei hohem Wellengang an das Smartphone
    \item Datenübertragung über große Entfernung
    \item Weltweiter Zugriff auf meine Daten 
    \item Anschaulich gestaltete App
\end{itemize}
Die Sender- und Empfängereinheit können bis zu 10 Kilometer voneinander entfernt stehen. 

