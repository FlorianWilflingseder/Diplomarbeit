Im Laufe dieser Diplomarbeit wurden sehr viele Erkenntnisse über die Programmiersprachen Dart und C gemacht. Es wurde C im Rahmen der Mirkoprozessor-Programmierung kenngelernt und Einblicke in eine Hardware nahe Entwicklung gewonnen, 
die in der Fachrichtung-Informatik normalerweise nicht üblich sind, der Aufbau neuer
\newline
Programmiersprachen kennengelernt und ein tiefergehendes Verständnis zur Entwicklung einer 
Full-Stack-Application wurde entwickelt. 
Dazu wurde der Umgang mit der Software Latex erlernt. 
Ebenso wurde bewusst wie wichtig ein Arbeitsplan und Zeit-Management sind um nicht die Zeit aus den Augen zu verlieren und mögliche Probleme die während der Entwicklungszeit auftreten können einplanen zu können.
\subsection*{Hürden}
Im Laufe der der Entwicklung eines Projektes gibt es immer wieder Hürden die überwunden werden müssen.
Dieses Projekt musste auch einige überwinden wie zum Beispiel defekte Sensoren mit schlechter Qualität aus China,
hohe Zoll-Preise und Versandkosten, lange Lieferzeiten bis zu 3 Monaten. Dazu kam, dass eine Funktion
(eine KI die verschiedene Wellengänge voneinander unterscheiden kann um keine falschen Alarme auszulösen)
aus den vorher genannten Gründen nicht umgesetzt werden konnte da das Risko die Sensoren oder die Chips durch Wasserschäden zu beschädigen.
Es wurden viele Libraries benötigt um einen Sensor auszulesen
und diese mussten zuerst verstanden werden.
Zusammenfassend können Probleme große Steine am Weg zum Ziel darstellen.